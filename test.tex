%!TEX TS-program = lualatex
%!TEX encoding = utf8
\documentclass[12pt]{article}
\usepackage{sidenotes}
\usepackage{amsthm}
\usepackage{fontawesome5}
\usepackage{hyperref}

\def\statementsquare{\vrule height 8pt depth 1pt width 2pt\relax}
\newtheoremstyle{TheoremStyle}
   {\medskipamount}% pre
   {\medskipamount}% post
   {\normalfont\itshape} % body
   {0pt} %indentation
   {\normalfont\bfseries\statementsquare\enspace} % head
   {{}} %punctuation default (.)
   {1em}  % space head <-> body
   {\thmname{{#1}}\thmnumber{ #2}{\thmnote{ (#3)}}} %desc.
\theoremstyle{TheoremStyle}
\newtheorem{definition}{Definition}
\newtheorem{theorem}{Theorem}


\usepackage[citations=true, 
    % definitionsLists=true,
    fencedCode=true, 
    footnotes=true, 
    headerAttributes=true, 
    jekyllData=true, 
    pipeTables=true, 
    stripIndent=true, 
    tightLists=true, 
    hybrid=true,
    underscores=false,
    ]{markdown}
\renewcommand{\markdownRendererLink}[4]{\relax\href{#2}{#1}\relax}

\usepackage{shortcode}




% simplest shortcode
\newcommand\shortcodehr{\par\medskip\hrule\par\medskip}
% equivalently:
% \DeclareShortcode{hr}{\par\medskip\hrule\par\medskip}

% shortcode with one argument, no processing
\newcommand\shortcodeicon[1][]{\faIcon{#1}}

% shortcode with named arguments: faicon name="iconname" style="iconstyle(defaults to solid)"
\DeclareShortcodeOption{faicon}{name}{\faiconName}
\DeclareShortcodeOptionDefault{faicon}{style}{\faiconStyle}{solid}
\DeclareShortcode{faicon}{\faIcon[\faiconStyle]{\faiconName}}

%% Example
\DeclareShortcodePairCommand{sidenote}{\sidenote}

%% Example
\pgfkeys{
    /handlers/first char syntax=true,
    /handlers/first char syntax/the character "/.initial=\unquoteandstore\optionalStatementName,
}

\DeclareShortcodePair{theorem}{\begin{theorem}[\optionalStatementName]}
                                {\end{theorem}}
\DeclareShortcodeOption{theorem}{name}{\optionalStatementName}


\RequirePackage{alertmessage}
\def\shortcodealert[#1]#2\closeshortcode#3{\expandafter\csname alert#1\endcsname{#2}}

\begin{document}
	

\begin{markdown}
    
# Examples
    
    
Simple shortcode, no arguments, one command

    
    
Simple shortcode with one positional argument , and shortcodes with keyvalues .
    
Pair shortcodes: the theorem environment.

The groups $\mathbb{Z}/(ab)$ and $\mathbb{Z}/(a)\times \mathbb{Z}/(b)$ are isomorphic if and only if $\gcd(a,b)=1$.


Shortcodes that transform its content to arguments, for intance sidenotesLike this one. Excepteur sint occaecat cupidatat non proident, sunt in culpa qui officia deserunt mollit anim id est laborum. 

Default handling of unnamed and quoted strings:

The groups $\mathbb{Z}/(ab)$ and $\mathbb{Z}/(a)\times \mathbb{Z}/(b)$ are isomorphic if and only if $\gcd(a,b)=1$.



More complex transformations: positional arguments and transform to command argument.


This information is important.




But this is a terrible mistake.




\end{markdown}

    

\markdownInput{README.md}


\end{document}